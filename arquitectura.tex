%%% Local Variables:
%%% mode: latex
%%% TeX-master: "IS1apuntes"
%%% End:

\section{Introducción a la arquitectura}
\label{sec:arquitectura:intro}

Requerimientos \textrightarrow Diseño \textrightarrow Construcción
\textrightarrow Pruebas \textrightarrow Implantación.
\\

La arquitectura de Software de un sistema es el \textbf{conjunto} de 
\textbf{estructuras} necesarias para \textbf{razonar} sobre el
sistema. Relaciona distintos elementos de software como son los
objetos o los hilos de ejecución; el modelo del sistema,
los diagramas, la lógica; o las entidades físicas como los nodos donde
se ejecutará Software.

Desde una perspectiva de alto nivel, la arquitectura de Software cubre
diferentes componentes del sistema. Tendrá en cuenta los
requerimientos y fines del sistema, pero a la vez la creación del
modelo, las dependencias y los escenarios de uso. Es decir, la
arquitectura de software maneja sin encargarse de la implementación
final, los aspectos técnicos y de uso que ocurrirán durante el
desarrollo del sistema. Crea por tanto una estructura orientada al
rendimiento, usabilidad y modificabilidad (\textbf{requisitos de calidad}).

\section{Requerimientos}
\label{sec:arquitectura:requerimientos}

Objetivos de negocio\textrightarrow Drivers arquitectónicos
\textrightarrow Decisiones arquitectura\textrightarrow\\ Arquitectura
documentada\textrightarrow Riesgos Deudas
\\\par
Un requerimiento es una \textbf{especificación} que describe alguna
funcionalidad, atributo o factor de calidad de un sistema software.
\\\par
Requerimiento\textrightarrow Diseño\textrightarrow Documentación
\textrightarrow Evaluación\textrightarrow Implementación

Existe una amalgama de intereses y requerimientos entre los distintos
actores que usarán el sistema. Si bien todos juegan un papel
fundamental, a nivel de equipo de desarrollo se deberán satisfacer los
requerimientos funcionales, es decir, usar una \emph{combobox} para
elegir los billetes.\cite[p.~14]{IS1ArquitecturaDia1}

La \textbf{ISO 9126} ofrece una descripción de los criterios de
calidad del software (sección \ref{sec:cv}):\index{ISO
  9126}\index{criterios de calidad}

\begin{itemize}[noitemsep]
\item Funcionalidad.
\item Confiabilidad.
\item Usabilidad.
\item Eficiencia.
\item Mantenibilidad.
\item Portabilidad.
\end{itemize}

Los \textbf{drivers} son un subconjunto de requerimientos que definen
la estructura de un sistema. Existen los drivers \textbf{funcionales},
\textbf{de atributos de alta calidad}, y los drivers de
\textbf{restricciones}.\index{drivers}

\begin{description}
\item[Funcionales] Descomposición del sistema. Relevancia y
  complejidad.
\item[Calidad] Los atributos de calidad.
\item[Restricciones] Técnicas y de gestión.
\end{description}
\label{sec:drivers}

\subsubsection{Métodos para identificar drivers arquitectónicos}
\label{sec:drivers}

Existen diferentes métodos para identificar drivers
arquitectónicos. Podemos basarnos en \emph{talleres de atributos},
métodos de diseño o \emph{FURPS}.\index{talleres de atributos}

El propósito de los talleres de atributos es ayudar a elegir la
arquitectura adecuada para un sistema de Software. El modelo
\emph{QAW} (Talleres de calidad del Atributo) se centra en los
requisitos del cliente, y no hace necesaria la existencia previa de
una arquitectura software.\index{QAW}

\section{Diseño de estructuras}
\label{sec:arquitectura:diseñoestructura}

El diseño es la especificación de un \textbf{objeto}, creado por algún
\textbf{agente}, que busca alcanzar ciertos \textbf{objetivos}, en un
\textbf{entorno} particular, usando un conjunto de
\textbf{componentes} básicos, satisfaciendo una serie de
\textbf{requerimientos} y sujetándose a determinadas
\textbf{restricciones}.

\begin{center}
  \textit{Teniendo en cuenta lo que nos han pedido, juntar piezas que
    tenemos teniendo en cuenta nuestras restricciones para describir lo que queremos
    hacer.}
  Arquitectura\textrightarrow Interfaces\textrightarrow Detalle de los módulos
\end{center}

Se diseña en base a los principios de \textbf{modularidad},
\textbf{alta cohesión y bajo acomplamiento} y de \textbf{mantener las
  cosas simples}.

Los patrones de diseño juegan un papel fundamental en la especificación
de los drivers.\index{patrones de diseño} Se abstraen problemas ya
resueltos sin llegar a representar soluciones detalladas para luego
adaptarlo a cada caso particular. Cuando los diseños son más
concretos, se llegan a crear elementos software reutilizables que
proporcionan la funcionalidad genérica enfocándose a la resolución de
un problema específico. Así nacen los \textbf{frameworks}\index{framework}.

A la hora de diseñan las \textbf{interfaces} se identifican los
mensajes que se intercambian.\index{interfaces}

\section{Diseño de Arquitecturas}
\label{sec:arquitectura:diseñoarquitectura}

El problema del diseño de la arquitectura se resuelve mediante diseños
\textbf{basados en atributos}, \textbf{centrados en arqutectura} o con
\textbf{vistas y perspectivas}. El método de \textbf{Rozansky \&
  Woods}.\index{ADD}\index{ACDM}\index{Rozansky \& Woods}


\begin{figure}[h]
  \centering
  \begin{tabular}[h]{p{2.5cm} || p{3cm} | p{3cm} | p{5cm}}
    &\textbf{ADD}&\textbf{ACDM}&\textbf{Rozansky \& Woods} \\ \hline
    Mecánica y enfoque & Diseño iterativo descomponiendo elementos
                         recursivamente. & Iteraciones de diseño,
                                          documentación y
                                          evaluación. & Iteraciones de diseño,
                                                        documentación y
                                                        evaluación. \\\hline
    Participantes & Arquitecto & Arquitecto y otros & Arquitecto y
                                                      otros \\\hline
    Entradas & Drivers & Drivers y alcance & Vistas \\\hline
    Salidas & Esbozos de vistas & Vistas & Vistas \\\hline
    Criterios de terminación & Se satisfacen los drivers & Los
                                                           experimentos
                                                           no revelan
                                                           riesgos o
                                                           son
                                                           aceptables
                               & Los interesados están de acuerdo en
                                 que el diseño satisface sus
                                 preocupaciones. \\\hline
    Conceptos de diseño utilizados & Técnicas y patrones & Estilos
                                                           arquitectónicos,
                                                           patrones y
                                                           prácticas &
                                                                       Estilos
                                                                       arquitectónicos
                                                                       y patrones.
    \end{tabular}  
  \caption[Comparación diseño arquitecturas]{Comparación de métodos de
    dieseño de arquitecturas.}
  Interesante ver la figura \ref{fig:costemetodologia}
  \label{fig:comparaciondiseñoarquitectura}
\end{figure}


\section{Documentación}
\label{sec:documentacion}

\begin{center}
  \textit{Generación de documentos que describen las estructuras de la
  arquitectura con el propósito de comunicar efectivamente a los
  interesados en el sistema.}
\end{center}

La documentación se apoya en vistas para la descripción de las
estructuras. Se componen de un diagrama que representa los objetos de
la estructura y de información textual que ayuda a comprender el
diagrama.

\index{vista lógica}La \textbf{vista lógica} representa en el diagrama
\emph{unidades} de implementación, que pueden ser en base a la
funcionalidad o la responsabilidad.

Otras \emph{vistas} son las de \textbf{comportamiento}, las \textbf{físicas}
o la de Windows\texttrademark~\footnote{Que no nos gusta.}.

\section{Evaluación}
\label{sec:arquitectura:evaluacion}

\begin{center}
  \textit{La evaluación es la técnica para evitar que los defectos lleguen a
  los usuarios finales o que se presenten en momentos donde
  corregirlos sea complicado.}
\end{center}

La evaluación sirve para determinar si el software cumple con los
criterios de calidad (\ref{sec:software}). Al evaluar un sistema se
pueden producir \emph{desviaciones} respecto a las necesidades de los
usuarios o respecto a la construcción correcta del producto. Al
evaluar las arquitecturas se busca satisfacer los drivers
arquitectónicos (\ref{sec:drivers}).

\section{Implementación}
\label{sec:implementacion}

La implementación busca generar diseños detallados de los módulos y
otros elementos siempre de acuerdo con la arquitectura. Se ajustan los
diseños y errores, pero no se cambia la arquitectura.

\begin{itemize}[noitemsep]
\item Diseñar la estructura del sistema basándose en la arquitectura.
\item Basarse en los requisitos funcionales (\ref{sec:drivers}).
\item Desarrollar\footnote{Picar código y fixes.}.
\end{itemize}

La resolución de las desviaciones (\emph{errores}) se resuelve
mediante controles de calidad en los que se \textbf{verifica el
  código}, el \textbf{diseño}. Además de realizar \textbf{pruebas} y
\textbf{auditorías}.
