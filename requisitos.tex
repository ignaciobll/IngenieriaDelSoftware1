%%% Local Variables:
%%% mode: latex
%%% TeX-master: "IS1apuntes"
%%% End:

La IR trata de los principios, métodos, técnicas y herramientas que
permiten descubrir, documentar y mantener los requisitos para sistemas
basados en computadora, de forma sistemática y repetible.

\section{Requisitos}

Definición formal de Jackson y Zave:

Todo problema software consiste en configurar una máquina para que ejerza unos efectos R en
un dominio K. La conexión de la máquina con el dominio se realiza a través de una interfaz S.

\begin{itemize}[noitemsep]
\item Los efectos R son los requisitos: Necesidades, metas,
  objetivos. Expresan ideas que no son una realidad ahora mismo, pero
  lo serán en un futuro, una vez que el sistema esté en
  funcionamiento.
\item El conocimiento del dominio K describe el contexto. Expresa
  realidades actuales, cosas que son verdad independientemente de la
  existencia o inexistencia del sistema.
\item La máquina (hw+sw) es la que realizará los requisitos R, gracias
  a S, que describe la conexión con el dominio K. Por tanto, S
  describe el comportamiento externo (observable) del sistema.
\item Idealmente: $K$ y $S \implies R$
\end{itemize}

Pirámide de \textbf{Leffingwell-Widrig}:
\begin{itemize}[noitemsep]
\item En primer lugar, las necesidades (needs), que son
objetivos a muy alto nivel del producto, y no debería
haber más de 3 ó 4.
\item Las features son una descripción breve, en lenguaje del
usuario, de las capacidades de alto nivel de un producto.
\item Finalmente, los requisitos software, en la base de la
pirámide, son descripciones más técnicas acerca de cómo
el producto realiza las features.
\end{itemize}


\subsubsection{Requisitos funcionales y no funcionales}
\label{sec:requisitos:funcionales-nofuncionales}

Los requisitos \textbf{funcionales} describen los servicios mientras que los
requisitos no funcionales son restricciones sobre los requisitos
funcionales.

Tipos de requisitos \textbf{no funcionales} (definiciones informales):
\begin{description}[noitemsep, align=right, labelwidth=2.5cm]
\item [Fiabilidad (reliability)] Es el grado con que un sistema cumple, o no, sus requisitos.
\item [Seguridad (security)] Hace referencia a la seguridad frente a ataques por parte de humanos o software malicioso.
\item [Seguridad (safety)] Hace referencia a la incapacidad del sistema de provocar accidentes.
\item [Usabilidad] Grado de integración del sistema en las tareas que realiza el usuario.
\item [Robustez] Capacidad del sistema de resistir situaciones extremas de todo tipo.
\item [Disponibilidad {\tiny (availability)}] Hace referencia al tiempo durante el cual el sistema no es utilizable.
\item [Rendimiento {\tiny (performance)}] Capacidad de hacer las tareas con el menor consumo de recursos posible.
\item [Tiempo de respuesta] Latencia entre una solicitud y la respuesta dada por el sistema.
\item [Capacidad] Cantidad de usuarios que pueden utilizar el sistema al mismo tiempo.
\item [Throughput] Cantidad de datos o transacciones procesadas por unidad de tiempo.
\item [Testabilidad] Capacidad de ser probado con facilidad.
\item [Portabilidad] Capacidad de ser transformado con facilidad a otra plataforma hw/sw.
\end{description}

\subsubsection{Proceso de Requisitos}
\label{sec:proceso-de-requisitos}

\subsubsection{Educción de Requisitos}
\label{sec:educcion}

Se orienta a la captura y descubrimiento de los requisitos. Para ello
se identifica a los interesados (stakeholders\footnote{Un stakeholders
  es toda aquella persona que se ve afectada por la existencia del
  futuro sistema}) y se establecen las primeras relaciones entre ellos
y el equipo de desarrollo.

Los requisitos pueden proceder de:
\begin{itemize}[noitemsep]
\item Metas.
\item Conocimiento del dominio de la aplicación. Interesados: Afectados por el cambio.
\item El entorno físico que rodea al sistema.
\item La organización.
\end{itemize}

Técnicas de educción:
\begin{itemize}[noitemsep]
\item Preguntas Libres de Contexto.
\item Brainstorming.
\item Creatividad.
\item Entrevistas.
\item Observación y análisis de tareas.
\item Casos de uso / Escenarios.
\item Prototipado.
\end{itemize}

\section{Análisis de Requisitos}
\label{sec:requisitos:analisis}

El Análisis de Requisitos consiste en detectar y resolver conflictos
entre los requisitos identificados, en precisar los límites del
sistema, precisar la interacción con su entorno, trasladar los
requisitos de usuario a requisitos implementables en software, etc. Se
realizan tres (sub)tareas fundamentales: \textbf{Clasificación, Modelización y
Negociación}.

\subsubsection{Clasificación}
\label{sec:requisitos:analisis:clasificacion}

Se clasifican en funcionales / no funcionales:
\begin{itemize}[noitemsep]
\item Por prioridades.
\item Por coste de implementación.
\item Por niveles (alto nivel, bajo nivel).
\item Según su volatilidad/estabilidad. Los volátiles son aquellos que
  posiblemente pierdan su razón de ser en el futuro, normalmente
  debido a cambios en políticas o en los procesos de gestión.
\item En requisitos sobre el proceso o sobre el producto.
\item Además, en Ingeniería de Sistemas, se realiza la ubicación de
  requisitos (Requirements Allocation), dónde se decide qué va a
  realizar el software y qué va a realizar el hardware.
\end{itemize}

\subsubsection{Modelización}
\label{sec:requisitos:analisis:modelizacion}

Ciertos aspectos de los requisitos se expresan mejor mediante modelos
de datos, de control, de estados, de interacción, de objetos, etc.

\subsubsection{Negociación}
\label{sec:requisitos:analisis:negociacion}

Es durante el análisis cuando se descubren muchos de esos
conflictos. EL CONFLICTO NO ES RECHAZABLE y no debe resolverse por
decreto, sino mediante un proceso de negociación. Desde este punto de
vista, los conflictos son positivos, pues SON FUENTE DE NUEVOS
REQUISITOS. Los acuerdos alcanzados deben ser convenientemente
anotados, favoreciéndose así la trazabilidad de los requisitos a sus
orígenes.

\section{Especificación de Requisitos}
\label{sec:requisitos:especificacion}

La especificación establece el almacenamiento de requisitos en algún
medio (electrónico o no) que permita su distribución, gestión,
impresión, búsquedas, etc.

El documento es el modo habitual de guardar y comunicar requisitos. Es
buena práctica utilizar, al menos, dos documentos, a distinto nivel de
detalle.

\begin{description}[noitemsep]
\item[DRU, Documento de Requisitos de Usuario (URD)] El DRU se escribe
  desde el punto de vista del usuario/cliente/interesado. Normalmente
  los requisitos de usuario, contenidos en la DRU, no poseen demasiado
  nivel de detalle (serían features, en su mayor parte). Se incluye la
  descripción del problema actual (razones por las que el sistema de
  trabajo actual es insatisfactorio) y las metas que se espera lograr
  con la construcción del nuevo sistema.
\item[ERS, Especificación de Requisitos de Software (SRS)] La ERS
  desarrolla mucho más los contenidos de la DRU. Los requisitos del
  software contenidos en la ERS son, por tanto, más detallados.
\end{description}

Llegado el momento de elaborar un documento en papel o, dicho de otra
forma, llegado el momento de volcar en papel un subconjunto de los
requisitos almacenados, de manera que las partes implicadas puedan
firmarlo a la manera de un contrato, hay varias formas de organizar
tal documento. Lo normal es seguir los estándares del IEEE.

Una ERS de calidad debería poseer, entre otros, los siguientes atributos:

\begin{itemize}[noitemsep]
\item No ambigua.
\item Completa.
\item Correcta.
\item Comprensible.
\item Verificable.
\item Consistente.
\item Independiente del diseño.
\item Anotada por importancia relativa.
\item Anotada por estabilidad relativa.
\item Anotada por versión.
\item Expresada a distintos niveles de abstracción.
\item Precisa.
\item Trazable.
\item Trazada.
\item Con referencias cruzadas.
\end{itemize}

\subsubsection{Validación de Requisitos}
\label{sec:requisitos:validacion}

El objetivo de la validación es descubrir problemas en el subconjunto
de requisitos que han sido seleccionados para su inclusión en la ERS,
antes de comprometer recursos a su implementación.

Durante la validación, el documento debe revisarse para descubrir
omisiones, conflictos, ambigüedades, y para comprobar su calidad y su
grado de adhesión a estándares. La fórmula más empleada para
validación son las llamadas Revisiones de Validación (Reviews)\index{Revisiones de Validación}. En
ellas, un grupo de personas (incluyendo usuarios) se ocupan de revisar
el documento de requisitos paraencontrar problemas. Estas reviews no
se hacen a ciegas, pues se suelen utilizar listas de comprobación
(checklists) que nos dicen qué es lo que hay que buscar.

\subsubsection{Gestión de Requisitos}
\label{sec:requisitos:gestion}

La Gestión de Requisitos consiste en gestionar los cambios a los
requisitos. Esto implica:

\begin{itemize}[noitemsep]
\item Definir procedimientos de cambios.
\item Cambiar los atributos de los requisitos afectados (prioridad, versión, etc.).
\item Mantener la trazabilidad.
\item Control de versiones del documento de requisitos.
\end{itemize}

Hay dos conceptos importantísimos en Gestión de Requisitos: la \textbf{trazabilidad} y las \textbf{líneas base}.\index{trazabilidad}

La trazabilidad hace referencia al establecimiento de enlaces entre
unas cosas y otras. En la Gestión de Requisitos se suelen manejar tres
tipos de trazabilidad:

\begin{description}[noitemsep]
\item [Trazabilidad hacia atrás] Dado un requisito, nos permite conocer
  su origen.
\item [Trazabilidad hacia delante] Dado un requisito, podemos conocer
  qué artefactos generados durante el proceso software se encuentran
  relacionados con él.
\item [Trazabilidad interna] Dado un requisito, la trazabilidad interna
  indica qué otros requisitos se encuentran relacionados con él.
\end{description}

Una línea base (\textbf{baseline}) es un conjunto de requisitos que, mediante acuerdo entre las partes
implicadas, se ha decidido no modificar.\index{baseline}
